\documentclass[11pt]{article}

\usepackage{amsmath,textcomp,amssymb,geometry,graphicx,enumerate,amsthm}
\usepackage{algpseudocode}
\usepackage[linesnumbered,ruled,vlined]{algorithm2e}
\usepackage{threeparttable, adjustbox, booktabs}

\def\endproofmark{$\Box$}
\newtheorem{theorem}{Theorem}[section]
\newtheorem{lemma}[theorem]{Lemma}
\theoremstyle{definition}
\newtheorem{definition}{Definition}[section]
\theoremstyle{remark}
\newtheorem*{remark}{Remark}
\newcommand{\indep}{\perp \!\!\! \perp}


\renewcommand\arraystretch{1.5}

%-----------------------------------------------------------------------------------

% Title information
\title{CS 188 HW 8 Challenge Question}
\author{Daniel Deng, SID 3034543526}
\pagestyle{myheadings}
\date{}

%-----------------------------------------------------------------------------------

\begin{document}
\maketitle

collaborators: none

\section{Decision Networks and HMMs}
\begin{enumerate}
\item[6.1)]
\begin{align*}
\begin{cases}
EU(theater) = (0.5)(100) + (0.5)(10) = 55 \\
EU(rent) = (0.5)(80) + (0.5)(40) = 60 \\
\end{cases} \\
\implies MEU(\emptyset) = 60 \\
\implies \underset{A}{\mathrm{arg}\!\max}\, EU(A) = rent \\
\end{align*}

\item[6.2)]
\begin{align*}
\text{Since F is unobserved, S and M are independent due to common effect} \\
\implies P(M|+s) = P(M) \\
\implies \begin{cases}
EU(theater|+s) = EU(theater) = 55 \\
EU(rent|+s) = EU(rent) = 60 \\
\end{cases} \\
\implies MEU(\{+s\}) = MEU(\{-s\})= 60 \\
\implies \text{Optimal action for both +s and -s} = rent \\
\implies VPI(S) = 60 -60 = 0
\end{align*}

\item[6.3)]
\begin{align*}
MEU(G) = \max_a \sum_{m} P(m|G)\,U(m,a)\\
EU(theater|+g) = (0.644)(100) + (0.356)(10) = 67.96 \\
EU(rent|+g) = (0.644)(80) + (0.356)(40) = 65.76 \\
\implies MEU(+g) = 67.96 \\
EU(theater|-g) = (0.293)(100) + (0.707)(10) = 36.37 \\
EU(rent|-g) = (0.293)(80)+(0.707)(40) = 51.72 \\
\implies MEU(-g) = 51.72 \\
\implies VPI(G) = (67.96)(0.59) + (51.72)(0.41) - 60 = 1.3016\\
\end{align*}

\item[6.4)]
\begin{enumerate}
\item[(i)]
\begin{itemize}
\item A is false. VPI is nonnegative.
\item B is true. $S \indep M$ when F is unobserved.
\item C is false. See above.
\item D is false.
\begin{align*}
MEU(+f) = 81.37,
MEU(-f) = 43.8 \\
VPI(F) = 65.5906 - 60 = 5.5906
\end{align*}
\item E is false. $VPI(G) = 1.3016$
\end{itemize}

\item[(ii)]
\begin{itemize}
\item A is false. VPI is nonnegative.
\item B is potentially true. For example. M-F might be a trivial edge.
\item C is potentially true. Observing G activates the path from S to M
\item D is potentially true. Depends on the CPT.
\item E is potentially true. Same reason as above.
\end{itemize}

\item[(iii)]
\begin{itemize}
\item A is false. VPI is nonnegative.
\item B is true. Observing F blocks the path from G to M.
\item C is false. See above.
\item D is false. $VPI(F)>0$.
\item E is false. $VPI(G)>0$.
\end{itemize}

\item[(iv)]
\begin{itemize}
\item A is false. See calculated value.
\item B is true.
\item C is false.
\item D is true.
\item E is false.
\end{itemize}
\end{enumerate}

\item[6.5)]
\begin{itemize}
\item A is equivalent. Marginalization + memoryless property.
\item B is equivalent. Same as A by memoryless property.
\item C is not equivalent. Need to eliminate $X_T$.
\item D is not equivalent. Same reason as above.
\end{itemize}

\item[6.6)] Since we want backward recursion,
\[
\beta(X_t) = \sum_{x_{t+1}} \beta (X_{t+1} = x_{t+1})P(X_{t+1} = x_{t+1} | X_t)P(E_{t+1} = e_{t+1} | X_{t+1} = x_{t+1})
\]
(CEBF)

\item[6.7)]
\begin{itemize}
\item A is false. Cannot eliminate $X_t$
\item B is false. Equals $P(X_t, E_{1:T})$ instead
\item C is true. Equivalent to $\frac{P(X_t, E_{1:T})}{P(E_{1:T})}$
\item D is false. Missing $E_{t+1:T}$
\end{itemize}
\end{enumerate}
\end{document}


