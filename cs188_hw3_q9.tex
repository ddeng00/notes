\documentclass[11pt]{article}

\usepackage{amsmath,textcomp,amssymb,geometry,graphicx,enumerate,amsthm}
\usepackage{algpseudocode}
\usepackage[linesnumbered,ruled,vlined]{algorithm2e}
\usepackage{threeparttable, adjustbox, booktabs}

\def\endproofmark{$\Box$}
\newtheorem{theorem}{Theorem}[section]
\newtheorem{lemma}[theorem]{Lemma}
\theoremstyle{definition}
\newtheorem{definition}{Definition}[section]
\theoremstyle{remark}
\newtheorem*{remark}{Remark}

\renewcommand\arraystretch{1.5}

%-----------------------------------------------------------------------------------

% Title information
\title{Challenge Q9 HW3}
\author{Daniel Deng, SID 3034543526}
\pagestyle{myheadings}
\date{}

%-----------------------------------------------------------------------------------

\begin{document}
\maketitle

Collaborators: None
\newpage
\section*{Q9. Multi-Agent Search}
\begin{enumerate}
\item[9.1)]
For one turn, the branching factor for PacAdv is $M=5$ and the branching factor for PacMax is $MN=15$. Assume we know nothing about the upper and lower bounds of the utility scores, \underline{the lower bound for pruning is 0} when $\alpha-\beta$ pruning is inconclusive about the leaf nodes given the current information on nodes, and \underline{the upper bound for pruning is $(M-1)(MN-1)=56$} because we need to look at the first intermediate branch completely to establish $\alpha$/$\beta$ values and  we need to at least look at one leaf node in all other intermediate branches before deciding to prune the remaining leaves.

\item[9.2)]
Following the logic in 9.1), the upper bound is $(6-1)(6* 10 - 1)=295$.

\item[9.3)]
One 1 leaf node (5 next to 7) will be pruned in alpha-beta pruning due to $(7) < (8)$.

\item[9.4)]
If the first and the third branch of the root minimizing node are switched,\\ \underline{the number of leaf nodes to be pruned increases to 4}. Since the swapped first branch evaluates to 3 first, leaf 5 next to 4 can be pruned, leaf 8 can be pruned (due to it being next to evaluated 6), and leaf 7 and 5 can be pruned all together.

\item[9.5)]
The list of possible values are $2,3,4,5,5,6,6,7,7,8,10$. Since the root node is minimizing, the largest value it can take is $7$ ($7,8,10$ as children), which rules out option $G$ and $H$. The smallest value the root node can take is $3$, because the minimizing combination for the middle maximizing node is $(2,3)$. Thus, option $A$ is ruled out. Therefore, the possible values are:
\begin{enumerate}
\item[B.] 3 is possible. An example sequence is $4,5,5,6,6,2,3,7,7,8,10$.
\item[C.] 4 is possible. An example sequence is $3,5,5,6,6,2,4,7,7,8,10$.
\item[D.] 5 is possible. An example sequence is $3,4,5,6,6,2,5,7,7,8,10$.
\item[E.] 6 is possible. An example sequence is $3,4,6,5,5,2,6,7,7,8,10$.
\item[F.] 7 is possible. An example sequence is $3,4,7,5,5,2,7,6,6,8,10$.
\end{enumerate}
\end{enumerate}


\end{document}


