\documentclass[11pt]{article}

\usepackage{amsmath,textcomp,amssymb,geometry,graphicx,enumerate,amsthm}
\usepackage{algpseudocode}
\usepackage[linesnumbered,ruled,vlined]{algorithm2e}
\usepackage{threeparttable, adjustbox, booktabs}
\usepackage{outlines}

\def\endproofmark{$\Box$}
\newtheorem{theorem}{Theorem}[section]
\newtheorem{lemma}[theorem]{Lemma}
\theoremstyle{definition}
\newtheorem{definition}{Definition}[section]
\theoremstyle{remark}
\newtheorem*{remark}{Remark}

\renewcommand\arraystretch{1.5}

%-----------------------------------------------------------------------------------

% Title information
\title{The Chinese Economy}
\author{Daniel Deng}
\pagestyle{myheadings}
\date{}

%-----------------------------------------------------------------------------------

\begin{document}
\maketitle

\section{Introduction}

\subsection{Growth Miracle}

China had a growth miracle for more than 40 years due to its successful reform strategy in the transition from a \textit{centrally planned economy} to a \textit{market economy} (currently an upper middle income country; better economic performance than other transition economies):
\begin{outline}[enumerate]
\1 GDP gap between China and the US has been reducing since 2006
	\2 Currently, GDP[China] $\approx$ 0.7 GDP[US] $\approx$ 0.2 GDP[World]
	\2 GDP per capita in China increased from the \$100s (1970s) to ~\$9,000 (current)
	\2 The GDP purchasing power parity PPP[China] $>$ PPP[US] (first occurred in 2014) and still growing
\1 \underline{Poverty alleviation} since 1978: 66.6 million people lived below poverty (1990) $\to$ 11.2 million (2010) $\to$ 1\% of all population (current)
\1 \underline{Industrialization}: 75\% labor force in agriculture (1978) $\to$ 35\% (2011)
\1 Like other East Asian countries, China emphasized integration in the world economy and aggressive export strategy: 1\% of world export (late 1970s) $\to$ 13\% (2015; \#1 export country in US dollars) $\to$ dominant trade partner for most countries in 2020
\end{outline}

\subsection{Problems}

\begin{outline}[enumerate]
\1 Economic reforms have stalled in the last 15 years with no major market reforms since the late 1990s $\to$ large inefficient state sector (44\% of assets, roughly 30\% of output)
\1 Inability to motivate more labor out of agriculture $\to$ declining labor force
\1 Large indebtedness of local governments
\1 Corruption (corruption perception index of 39 in 2018)
	\2 Decline in control during Hu Jintao years but has since gone up due to Xi Jinping's anti-corruption campaign
\1 No progress in political liberalization; tightening of repression under Xi Jinping with uncertainty about succession rules
	\2 Lack of institutional convergence (in spite of increased government capacity and effectiveness) towards advanced capitalist societies; no desire to introduce institutions of "rule of law" nor accountable, democratic forms of government 
	\2 Autocratic political institutions (-8 polity score)
\end{outline}
\clearpage

\section{Geographical Background}
Geography is important for understanding economic development:
\begin{outline}[enumerate]
\1 Traditional subsistence economy affected by climate, bacteria, land, water, etc.
\1 Modern economy affected by trade, coastal lines, ports, roads, etc.
\1 According to Jared Diamond, in \textit{Guns, Germs, and Steel}, Eurasian landmass saw more economic development due to access to more domesticable species and it being a horizontal stretch (agricultural knowledge can spread and be applicable to in many different regions)  
\end{outline}

\subsection{Geography of China}
China has the third biggest landmass, after Russia and Canada, but largely rugged and inhospitable (14\% arable land). 
\begin{outline}[enumerate]
\1 Due to difficulty accessing the coast (coast found only in the East, which is mostly rugged and hilly), Chinese civilization traditionally has an inward orientation, occurring near the major rivers
\1 Great difference in climate across regions
\1 Large population density (94\%) to the east of the Aihui-Tengchong line.
\end{outline}

\begin{table}[ht]
\caption{Macroregions of China}
\begin{adjustbox}{width={\textwidth},center}
\begin{tabular}{lc}
\toprule
&Characteristic(s)\\
\midrule
North & dry; dense population, agriculture, and industry\\
Northeast& dense heavy industry\\
Upper Yangtze & flat, continental; dense population; environmental and water management problems\\
Middle Yangtze & dense industry and rich communication\\
Lower Yangtze & richest region\\
Coast& relatively isolated due to mountain ranges\\
Far-South & prosperous region that did not play a big role in Chinese history\\
Southwest & mountainous; minority ethnic groups\\
\bottomrule
\end{tabular}
\end{adjustbox}
\end{table}

\subsection{Provinces of China}
China has 22 provinces (and Taiwan), including \underline{4 national municipalities} (Beijing, Shanghai, Chongqing, and Tianjin), \underline{5 autonomous regions} (Guangxi, Tibet, Ningxia, Inner Mongolia, and Xinjiang), and \underline{2 special autonomous regions} (Hong Kong and Macau).

\subsection{Natural Resources in China}
China has oil, coal (mostly in the North), and shale gas deposits, as well as substantial hydroelectric potential; however, there is a scarcity of natural resources relative to the size of the population and the economy $\to$ dependency on imports
\clearpage

\section{Historical Background}
\subsection{Before 1840 (Pre-Modern)}
The Shang dynasty (1600-1046 BCE) is the earliest Chinese dynasty on which we have historical information; many features of Chinese institutions through history are already present during the Shang dynasty:
\begin{outline}[enumerate]
\1 Northern China, near the Yellow River, was the center of early Chinese civilization
\1 East-west flow of rivers + similar agricultural conditions $\to$ homogeneous economic conditions $\to$ less emphasis on trade but more emphasis on collective work and specialization
	\2 Agrarian society without private property of land
	\2 Strong specialization in artisanal work (pottery, bronze, etc.)
	\2 Strong cultural homogeneity and relative isolation from outside (except frictions with Northern nomads)
\end{outline}

\subsubsection{Harmony, Hierarchy, and Ancestor Worship}
The sense of harmony was linked with hierarchy (ranking), and the general optimism of the time led to a benevolent view of the emperor. 

In addition, hierarchy among among ones ancestors affected their own social hierarchy, and ancestors are ranked higher than the living and the unborn $\to$ strong ancestor worship legitimized imperial regime and bureaucracy

\subsubsection{Society}
Early Chinese cities were associated with ruling clans (with internal hierarchy based on patrilineal kinship) whose chiefs were directly responsible to the Emperor
\begin{outline}[enumerate]
\1 Hierarchical distribution of resources diminished value of commercial activities such as trading; however, large clans still played a strong role in the economy
\1 Underdeveloped class/caste systems
\end{outline}

\subsubsection{Culture and Beliefs}
\paragraph{Confucianism}
Confucianism, developed during the Zhou dynasty, formalized existing cultural norms (filial piety; hierarchy within family and society; duties of sons, fathers, and emperors; importance of benevolence and rites; optimism that humans can be good and be improved; and, nobility of virtue instead of blood). It embraced large parts of what is known as collective culture:
\begin{outline}[enumerate]
\1 Humans are social beings defined in relation to others;
\1 The world is stable but individuals are infinitely malleable;
\1 Prestige is associated to how well one adapts to the society.
\end{outline}

\paragraph{Legalism}
Legalism, propagated by the Qin dynasty (221-206 BCE), emphasized "rule by law" instead of "rule of law" $\to$ law defines lists of strict punishments for particular behaviors to submit people to the Emperor and the state.

\begin{remark}
Chinese history has oscillated between Confucianism and legalism.
\end{remark}

\paragraph{Meritocracy}
The imperial exam system was used as a recruitment tool for bureaucrats to assure competence and merit in promotion (quality of government administration) $\to$ relatively low social stratification and high social mobility
\begin{outline}[enumerate]
\1 Started during the Han dynasty (206 BCE - 220 CE), generalized during the Tang dynasty (618-907), and abolished in 1905
\1 Content of the exams promoted conservative thinking
\end{outline}

\subsubsection{State Organization}
Starting from the Qin dynasty, subnational government officials were designed by the Emperor as a tool of strategic control over the territory $\to$ because of distance and bad communications, they had a lot of autonomy, which necessitated self-contained administrative structures $\to$ central administrative departments were replicated at the local level.

\subsubsection{Notable Economic Factors}
\begin{outline}[enumerate]
\1 During the Song dynasty (northern 960-1126; southern 1127-1279), China had the highest per capita production and income in the world, but has since declined relative to continental Europe
	\2 China's total GDP possibly still the largest in the world during the Qin dynasty
	\2 Scarcity of animals in agriculture compared to Europe
	\2 High productivity of land, but not of labor
\1 Sophisticated system of waterways (the grand canal)
\1 Population grew from 72 million in 1400 to 310 million in 1800 and political stability assured stable incomes.
\end{outline}

\subsection{1840-1911 (Pre-Republic)}
China paid reparations in all the wars it lost and the Qing dynasty collapsed in 1911; opening to the outside led to national humiliation, and up to 80 treaty ports were open at some point
\begin{outline}[enumerate]
\1 Britain was unsatisfied with the large flow of silver into China due to Chinese trade surplus $\to$ exported opium to China through Hong Kong to balance trade $\to$ Chinese prohibition of opium $\to$ British aggression and the Opium War of 1840 $\to$ \underline{Treaty of Nanking} (1842) conceded Hong Kong, opened first 5 treaty ports, and allowed extraterritorial rights in Chinese territories (foreign exemption from Chinese domestic laws)
	\2 Many other unequal treaties ensued with Germany, France, and Japan
\1 \underline{Taiping rebellion} (1850-1864; more than 20 million dead) significantly weakened the Qing dynasty, who had to rely on outside support in crushing the rebellion
\1 \underline{Sino-Japanese war of 1895} $\to$ Japan seized Taiwan
\1 \underline{Boxer rebellion} $\to$ further weakened the imperial government
\end{outline}

\subsection{1911-1949 (Republic)}
\begin{table}[ht]
\centering
\caption{Key events during Republic China.}
\begin{tabular}[t]{lc}
\toprule
&Description\\
\midrule
1911&End of the Qing dynasty and beginning of the Republic of China\\
1911-27&Warlords divided China\\
1927-39&Ten years of relative peace under KMT rule\\
1937-45&Sino-Japanese War\\
1945-49&Civil war\\
\bottomrule
\end{tabular}
\end{table}

\subsubsection{Patterns of Industrialization (1912-1937)}
South of the Great Wall, industrialization concentrated in a few treaty ports (Shanghai, Tianjin, and Qingdao), producing textile and food products for domestic markets (firms were started by foreigners but Chinese capitalists soon became a major force). On the other hand, in the Northeast, industrialization was sponsored by the Japanese government, with heavy industry and railroads to exploit coal and iron ore for the Japanese market.
\begin{table}[ht]
\centering
\caption{Two patterns of industrialization.}
\begin{tabular}[t]{lcc}
\toprule
&China proper&Manchuria\\
\midrule
Market&Domestic China&Japanese industry\\
Ownership&Chinese, foreign&Foreign\\
Structure&Light, consumer goods&Heavy, mining, producer goods\\
Skill formation&Steady accumulation&Little transfer of skills\\
Linkages&Backward&Few or none\\
\bottomrule
\end{tabular}
\end{table}%

\subsection{1949-1978 (People's Republic, Pre-Reform)}
\begin{table}[ht]
\centering
\caption{Key events during People's Republic of China, pre-reform.}
\begin{tabular}[t]{lc}
\toprule
&Description\\
\midrule
1949-52&Economic recovery to 1937 level\\
1953-57&Copy of the Soviet central planning model\\
1958-59&Great Leap Forward\\
1960-62&Worst famine in human history\\
1962-65&Economic recovery\\
1966-76&Cultural Revolution\\
1977-1978&Transitional period\\
\bottomrule
\end{tabular}
\end{table}

\begin{remark}
Less than 10 years of normal socialist development.
\end{remark}

\subsubsection{Initial Economic Conditions}
While there are heavy industries based in the Northeast (inherited from the Japanese) and light industries in a few port cities, on average China has a very poor peasant economy, with major industrialization indicators even lower than India $\to$ China's total GDP at lowest in its history

Hisotrical Perspective of China's Economic Position in the world
\begin{outline}[enumerate]
\1 1978 AD (at the inception of the reform): China's per capita GDP very low
\end{outline}

\subsubsection{Development Strategy}
Like the Soviet Union, China adopted a Big Push strategy that prioritized heavy industry development $\to$ high investment rate ($> 30\%$), mostly in heavy industry/producer goods ($> 80\%$) $\to$ rapid industrial growth rate ($>10\%$ per annum)

\begin{table}[ht]
\centering
\caption{Comparisons of mainland China with Taiwan and Hong Kong.}
\begin{tabular}[t]{lcc}
\toprule
&Mainland&Taiwan/Hong Kong\\
\midrule
Priority&Heavy industry&Light industry\\
Strategy Key&Upstream industries&Consumer goods industries\\
Demand&Domestic industries&Domestic/export markets\\
Investment&Government&Private and government\\
Savings&Government&Private and government\\
Household&Slow income growth&Moderate income growth\\
Coordination&Plan&Market\\
\bottomrule
\end{tabular}
\end{table}%

\subsubsection{Problems}
\paragraph{Slow Individual Consumption Growth}
TODO

\paragraph{Slow Employment Creation}
While there was a net increase of 191 million (207 million to 398 million) total labor force from 1952 to 1978, employment in non-agriculture sectors only had a net increase of 71 million (34 million to 105 million) $\to$ employment creation only absorbed 37\% of the increase in labor force
\begin{outline}[enumerate] $\to$ strong decline in labor productivity
\1 The agricultural labor force was 70\% larger in 1978 than in 1952, while cultivated land remained constant
\1 Heavy industry is capital intensive but not labor intensive, while China was scarce in capital but abundant in labor $\to$ 5.4\% labor force growth vs. 11\% output growth in industry sector
\end{outline}

\subsection{1979-Present (People's Republic, Post-Reform)}

\section{China under Central Planning}

\section{A Picture of the Reform Process}

\section{China's Specific Institutions}

\section{Growth and Structural Change}

\section{Agriculture}

\section{State Industry and State-Owned Enterprises}

\section{Non-State Industry}

\section{The Financial Sector}

\section{Population, Labor, and Income Distribution}

\section{Foreign Trade and Investment}

\section{China in the Twenty-First Century}
\end{document}


