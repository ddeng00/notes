\documentclass[11pt]{article}

\usepackage{amsmath,textcomp,amssymb,geometry,graphicx,enumerate,amsthm}
\usepackage{algpseudocode}
\usepackage[linesnumbered,ruled,vlined]{algorithm2e}
\usepackage{threeparttable, adjustbox, booktabs}
\usepackage{outlines}

\def\endproofmark{$\Box$}
\newtheorem{theorem}{Theorem}[section]
\newtheorem{lemma}[theorem]{Lemma}
\theoremstyle{definition}
\newtheorem{definition}{Definition}[section]
\theoremstyle{remark}
\newtheorem*{remark}{Remark}

\renewcommand\arraystretch{1.5}

%-----------------------------------------------------------------------------------

% Title information
\title{The Chinese Economy}
\author{Daniel Deng}
\pagestyle{myheadings}
\date{}

%-----------------------------------------------------------------------------------

\begin{document}
\maketitle

\section{Introduction}

\subsection{Growth Miracle}

China had a growth miracle for more than 40 years due to its successful reform strategy in the transition from a \textit{centrally planned economy} to a \textit{market economy} (currently an upper middle income country; better economic performance than other transition economies):
\begin{outline}[enumerate]
\1 GDP gap between China and the US has been reducing since 2006
	\2 Currently, GDP[China] $\approx$ 0.7 GDP[US] $\approx$ 0.2 GDP[World]
	\2 GDP per capita in China increased from the \$100s (1970s) to ~\$9,000 (current)
	\2 The GDP purchasing power parity PPP[China] $>$ PPP[US] (first occurred in 2014) and still growing
\1 \underline{Poverty alleviation} since 1978: 66.6 million people lived below poverty (1990) $\to$ 11.2 million (2010) $\to$ 1\% of all population (current)
\1 \underline{Industrialization}: 75\% labor force in agriculture (1978) $\to$ 35\% (2011)
\1 Like other East Asian countries, China emphasized integration in the world economy and aggressive export strategy: 1\% of world export (late 1970s) $\to$ 13\% (2015; \#1 export country in US dollars) $\to$ dominant trade partner for most countries in 2020
\end{outline}

\subsection{Problems}

\begin{outline}[enumerate]
\1 Economic reforms have stalled in the last 15 years with no major market reforms since the late 1990s $\to$ large inefficient state sector (44\% of assets, roughly 30\% of output)
\1 Inability to motivate more labor out of agriculture $\to$ declining labor force
\1 Large indebtedness of local governments
\1 Corruption (corruption perception index of 39 in 2018)
	\2 Decline in control during Hu Jintao years but has since gone up due to Xi Jinping's anti-corruption campaign
\1 No progress in political liberalization; tightening of repression under Xi Jinping with uncertainty about succession rules
	\2 Lack of institutional convergence (in spite of increased government capacity and effectiveness) towards advanced capitalist societies; no desire to introduce institutions of "rule of law" nor accountable, democratic forms of government 
	\2 Autocratic political institutions (-8 polity score)
\end{outline}
\clearpage

\section{Geographical Background}
Geography is important for understanding economic development:
\begin{outline}[enumerate]
\1 Traditional subsistence economy affected by climate, bacteria, land, water, etc.
\1 Modern economy affected by trade, coastal lines, ports, roads, etc.
\1 According to Jared Diamond, in \textit{Guns, Germs, and Steel}, Eurasian landmass saw more economic development due to access to more domesticable species and it being a horizontal stretch (agricultural knowledge can spread and be applicable to in many different regions)  
\end{outline}

\subsection{Geography of China}
China has the third biggest landmass, after Russia and Canada, but largely rugged and inhospitable (14\% arable land). 
\begin{outline}[enumerate]
\1 Due to difficulty accessing the coast (coast found only in the East, which is mostly rugged and hilly), Chinese civilization traditionally has an inward orientation, occurring near the major rivers
\1 Great difference in climate across regions
\1 Large population density (94\%) to the east of the Aihui-Tengchong line.
\end{outline}

\begin{table}[ht]
\caption{Macroregions of China}
\begin{adjustbox}{width={\textwidth},center}
\begin{tabular}{lc}
\toprule
&Characteristic(s)\\
\midrule
North & dry; dense population, agriculture, and industry\\
Northeast& dense heavy industry\\
Upper Yangtze & flat, continental; dense population; environmental and water management problems\\
Middle Yangtze & dense industry and rich communication\\
Lower Yangtze & richest region\\
Coast& relatively isolated due to mountain ranges\\
Far-South & prosperous region that did not play a big role in Chinese history\\
Southwest & mountainous; minority ethnic groups\\
\bottomrule
\end{tabular}
\end{adjustbox}
\end{table}

\subsection{Provinces of China}
China has 22 provinces (and Taiwan), including \underline{4 national municipalities} (Beijing, Shanghai, Chongqing, and Tianjin), \underline{5 autonomous regions} (Guangxi, Tibet, Ningxia, Inner Mongolia, and Xinjiang), and \underline{2 special autonomous regions} (Hong Kong and Macau).

\subsection{Natural Resources in China}
China has oil, coal (mostly in the North), and shale gas deposits, as well as substantial hydroelectric potential; however, there is a scarcity of natural resources relative to the size of the population and the economy $\to$ dependency on imports
\clearpage

\section{Historical Background}
\subsection{Before 1840 (Pre-Modern)}
The Shang dynasty (1600-1046 BCE) is the earliest Chinese dynasty on which we have historical information; many features of Chinese institutions through history are already present during the Shang dynasty:
\begin{outline}[enumerate]
\1 Northern China, near the Yellow River, was the center of early Chinese civilization
\1 East-west flow of rivers + similar agricultural conditions $\to$ homogeneous economic conditions $\to$ less emphasis on trade but more emphasis on collective work and specialization
	\2 Agrarian society without private property of land
	\2 Strong specialization in artisanal work (pottery, bronze, etc.)
	\2 Strong cultural homogeneity and relative isolation from outside (except frictions with Northern nomads)
\end{outline}

\subsubsection{Harmony, Hierarchy, and Ancestor Worship}
The sense of harmony was linked with hierarchy (ranking), and the general optimism of the time led to a benevolent view of the emperor. 

In addition, hierarchy among among ones ancestors affected their own social hierarchy, and ancestors are ranked higher than the living and the unborn $\to$ strong ancestor worship legitimized imperial regime and bureaucracy

\subsubsection{Society}
Early Chinese cities were associated with ruling clans (with internal hierarchy based on patrilineal kinship) whose chiefs were directly responsible to the Emperor
\begin{outline}[enumerate]
\1 Hierarchical distribution of resources diminished value of commercial activities such as trading; however, large clans still played a strong role in the economy
\1 Underdeveloped class/caste systems
\end{outline}

\subsubsection{Culture and Beliefs}
\paragraph{Confucianism}
Confucianism, developed during the Zhou dynasty, formalized existing cultural norms (filial piety; hierarchy within family and society; duties of sons, fathers, and emperors; importance of benevolence and rites; optimism that humans can be good and be improved; and, nobility of virtue instead of blood). It embraced large parts of what is known as collective culture:
\begin{outline}[enumerate]
\1 Humans are social beings defined in relation to others;
\1 The world is stable but individuals are infinitely malleable;
\1 Prestige is associated to how well one adapts to the society.
\end{outline}

\paragraph{Legalism}
Legalism, propagated by the Qin dynasty (221-206 BCE), emphasized "rule by law" instead of "rule of law" $\to$ law defines lists of strict punishments for particular behaviors to submit people to the Emperor and the state.

\begin{remark}
Chinese history has oscillated between Confucianism and legalism.
\end{remark}

\paragraph{Meritocracy}
The imperial exam system was used as a recruitment tool for bureaucrats to assure competence and merit in promotion (quality of government administration) $\to$ relatively low social stratification and high social mobility
\begin{outline}[enumerate]
\1 Started during the Han dynasty (206 BCE - 220 CE), generalized during the Tang dynasty (618-907), and abolished in 1905
\1 Content of the exams promoted conservative thinking
\end{outline}

\subsubsection{State Organization}
Starting from the Qin dynasty, subnational government officials were designed by the Emperor as a tool of strategic control over the territory $\to$ because of distance and bad communications, they had a lot of autonomy, which necessitated self-contained administrative structures $\to$ central administrative departments were replicated at the local level.

\subsubsection{Notable Economic Factors}
\begin{outline}[enumerate]
\1 During the Song dynasty (northern 960-1126; southern 1127-1279), China had the highest per capita production and income in the world, but has since declined relative to continental Europe
	\2 China's total GDP possibly still the largest in the world during the Qin dynasty
	\2 Scarcity of animals in agriculture compared to Europe
	\2 High productivity of land, but not of labor
\1 Sophisticated system of waterways (the grand canal)
\1 Population grew from 72 million in 1400 to 310 million in 1800 and political stability assured stable incomes.
\end{outline}

\subsection{1840-1911 (Pre-Republic)}
China paid reparations in all the wars it lost and the Qing dynasty collapsed in 1911; opening to the outside led to national humiliation, and up to 80 treaty ports were open at some point
\begin{outline}[enumerate]
\1 Britain was unsatisfied with the large flow of silver into China due to Chinese trade surplus $\to$ exported opium to China through Hong Kong to balance trade $\to$ Chinese prohibition of opium $\to$ British aggression and the Opium War of 1840 $\to$ \underline{Treaty of Nanking} (1842) conceded Hong Kong, opened first 5 treaty ports, and allowed extraterritorial rights in Chinese territories (foreign exemption from Chinese domestic laws)
	\2 Many other unequal treaties ensued with Germany, France, and Japan
\1 \underline{Taiping rebellion} (1850-1864; more than 20 million dead) significantly weakened the Qing dynasty, who had to rely on outside support in crushing the rebellion
\1 \underline{Sino-Japanese war of 1895} $\to$ Japan seized Taiwan
\1 \underline{Boxer rebellion} $\to$ further weakened the imperial government
\end{outline}

\subsection{1911-1949 (Republic)}
\begin{table}[ht]
\centering
\caption{Key events during Republic China.}
\begin{tabular}[t]{lc}
\toprule
&Description\\
\midrule
1911&End of the Qing dynasty and beginning of the Republic of China\\
1911-27&Warlords divided China\\
1927-39&Ten years of relative peace under KMT rule\\
1937-45&Sino-Japanese War\\
1945-49&Civil war\\
\bottomrule
\end{tabular}
\end{table}

\subsubsection{Patterns of Industrialization (1912-1937)}
South of the Great Wall, industrialization concentrated in a few treaty ports (Shanghai, Tianjin, and Qingdao), producing textile and food products for domestic markets (firms were started by foreigners but Chinese capitalists soon became a major force). On the other hand, in the Northeast, industrialization was sponsored by the Japanese government, with heavy industry and railroads to exploit coal and iron ore for the Japanese market.
\begin{table}[ht]
\centering
\caption{Two patterns of industrialization.}
\begin{tabular}[t]{lcc}
\toprule
&China proper&Manchuria\\
\midrule
Market&Domestic China&Japanese industry\\
Ownership&Chinese, foreign&Foreign\\
Structure&Light, consumer goods&Heavy, mining, producer goods\\
Skill formation&Steady accumulation&Little transfer of skills\\
Linkages&Backward&Few or none\\
\bottomrule
\end{tabular}
\end{table}%

\subsection{1949-1978 (People's Republic, Pre-Reform)}
\begin{table}[ht]
\centering
\caption{Key events during People's Republic of China, pre-reform.}
\begin{tabular}[t]{lc}
\toprule
&Description\\
\midrule
1949-52&Economic recovery to 1937 level\\
1953-57&Copy of the Soviet central planning model\\
1958-59&Great Leap Forward\\
1960-62&Worst famine in human history\\
1962-65&Economic recovery\\
1966-76&Cultural Revolution\\
1977-1978&Transitional period\\
\bottomrule
\end{tabular}
\end{table}

\begin{remark}
Less than 10 years of normal socialist development.
\end{remark}

\subsubsection{Initial Economic Conditions}
While there are heavy industries based in the Northeast (inherited from the Japanese) and light industries in a few port cities, on average China has a very poor peasant economy, with major industrialization indicators even lower than India $\to$ China's total GDP at lowest in its history

\subsubsection{Development Strategy}
Like the Soviet Union, China adopted a Big Push strategy that prioritized heavy industry development $\to$ high investment rate ($> 30\%$), mostly in heavy industry/producer goods ($> 80\%$) $\to$ rapid industrial growth rate ($>10\%$ per annum) + rapid structural change (\% GDP)

\begin{table}[ht]
\centering
\caption{Comparisons of mainland China with Taiwan and Hong Kong.}
\begin{tabular}[t]{lcc}
\toprule
&Mainland&Taiwan/Hong Kong\\
\midrule
Priority&Heavy industry&Light industry\\
Strategy Key&Upstream industries&Consumer goods industries\\
Demand&Domestic industries&Domestic/export markets\\
Investment&Government&Private and government\\
Savings&Government&Private and government\\
Household&Slow income growth&Moderate income growth\\
Coordination&Plan&Market\\
\bottomrule
\end{tabular}
\end{table}%

\subsubsection{Problems}
\paragraph{Slow Employment Creation}
While there was a net increase of 191 million (207 million to 398 million) total labor force from 1952 to 1978, employment in non-agriculture sectors only had a net increase of 71 million (34 million to 105 million) $\to$ employment creation only absorbed 37\% of the increase in labor force
\begin{outline}[enumerate] $\to$ strong decline in labor productivity
\1 The agricultural labor force was 70\% larger in 1978 than in 1952, while cultivated land remained constant
\1 Heavy industry is capital intensive but not labor intensive, while China was scarce in capital but abundant in labor $\to$ 5.4\% labor force growth vs. 11\% output growth in industry sector
\end{outline}
\clearpage

\section{China under Central Planning}
\subsection{Three Features of Socialist Systems}
\paragraph{The Party-State} The communist party is in power monopoly (controls everything and everyone), which is enshrined in the constitution, under the socialist system.
\begin{outline}[enumerate]
\1 Party jurisdictions:
	\2 Personnel (appointment, promotion, dismissal)
	\2 Ideology and propaganda (all media)
	\2 Mass organizations (unions, students, women, etc.)
	\2 Major decisions of the government (e.g., the five year plan)
\1 State/government jurisdictions:
	\2 Formulation of plans
	\2 Implementation of plans
	\2 Control over property, finance, prices, etc.
\end{outline}

\paragraph{Public Ownership} Public ownership dominates private ownership under the socialist system.
\begin{outline}[enumerate]
\1 Types of properties:
	\2 Means of production---for producing products or services
	\2 Personal property---for personal use
\1 Property rights---rights to generated income; rights of use; rights of transfer
	\2 State ownership is the highest form of public ownership, under which the government exercises all property rights (despite technically owned by the "people")
	\2 Collective/cooperative ownership is a lower form of public ownership; normally, the ownership belongs to a group of people, but, in reality, the local government controls and intervenes
\end{outline}

\paragraph{Central Planning} Central planning is the main mechanism of resource allocation under the socialist system.
\begin{outline}[enumerate]
\1 Government-controlled investment---five-year and annual plans determine investment priorities and the Politburo approves specific investment projects
	\2 Allocation of necessary resources to key projects (e.g., funds, labor, foreign exchange) $\to$ a "resource mobilization economy" (similar to war-time economies)
	\2 A politicized process that serves not just economic objectives
\1 Price determination---prices do no reflect supply and demand, but reflect the social necessary cost (excluding land and capital) $\to$ stable prices
	\2 Price policies channel funds to the government and achieve income redistribution
	\2 Low basic consumer needs (e.g., food, rent, health care) and factor (e.g., wages, interest rates, exchange rates) prices
	\2 High manufactured and luxury goods prices
\1 Material balance planning---balances sum of supplies and sum of demands
	\2 An output target for each producer + a supply plan for transfer of resources between producers + a schedule of usage coefficients linking inputs and outputs
	\2 Difficulty disaggregating the plan and gathering information, which involves a huge number of bureaucrats $\to$ feasible plan is possible, but likely inefficient
	\2 Incentives for managers to fulfill the plan (but not necessarily the actual need) $\to$ distortion of the output mix, which affects the input mix downstream
		\3 managers also bargain with planner to lower output quotas and to raise input supplies
\end{outline}

\begin{remark}
Managers under central planning have parallels with those in public corporations in that they have conflict of interests, albeit with different incentives.
\end{remark}

\subsection{Chinese Planning Compared to Soviet Planning}
\begin{outline}[enumerate]
\1 Different initial conditions:
	\2 Huge number of peasants
	\2 Scarce land
	\2 Labor-intensive agriculture technology
	\2 Lack of resources
	\2 Lower stage of development
\1 Mao's idea:
	\2 Rejection of the bureaucratic soviet model
	\2 No understanding of modern economy (plan or market)
	\2 Believed in political mobilization as a mean of problem solving
\end{outline}

\begin{remark}
At the end of their respective planning economy, China was much poorer than Central and Eastern Europe (low income level vs. upper-middle income level)
\end{remark}

\subsubsection{Three Major Differences}
\paragraph{Regional Decentralization} 
\begin{outline}[enumerate]
\1 USSR U-form (centralized coordination; specialization; need to manage information flow) vs. Chinese M-form (localized coordination; duplication; unscalable; helped transition to Market economy) organizations.
	\2 The Soviet Union had a small number of very large industrial firms, whereas China had a larger number of industrial firms with smaller size. 
\1 1958 decentralization $\to$ economic disasters $\to$ re-centralization in 1962-65
	\2 Delegation of state-owned enterprises to local governments $\to$ central government controlled enterprises: 9,300 (1957) $\to$ 1,200 (1958)
	\2 Material balancing at the local level
	\2 Allocation of local government budgets and allowed local governments to make investment decisions
	\2 People's commune
\1 1970 decentralization $\to$ incomplete re-centralization in 1975
	\2 Mao criticized the Soviet Union $\to$ perceived danger of invasion $\to$ moved industry inland (Third Front campaign)
	\2 Provincial governments set up independent industrial bases to diversify risks from war
	\2 Most SOEs delegated to local governments, and more than 50\% of local investment when to steel and iron
	\2 Establishment of rural small industry to support "agricultural mechanization" 
\end{outline}
\begin{remark}
Regional decentralization reflected both the Chinese tradition of decentralization and Mao's fear of war.
\end{remark}

\paragraph{Urban-Rural Divide}
\begin{outline}[enumerate]
\1 \textit{Danwei} ("work unit")---microcosm of urban society, into which individuals were born, lived, worked, and died
	\2 Provided most welfare benefits and served as a hub of social networks
	\2 Exercised political control $\to$ control over one's personnel file $\to$ ownership over labor $\to$ barrier to labor mobility

\1 Urban residents
	\2 Better access to essential consumer goods and other scarce goods; primary and secondary education; guaranteed job
	\2 Urban sector working benefits---health care; retirement benefits; low-cost housing

\1 Rural residents
	\2 \textit{Danwei} = resident village within a Commune $\to$ relied on support
	\2 Limited ways to obtain an urban residence permit: obtain a job in urban sector or marry an urban person $\to$ without an urban residence permit, rural residents would not have access to food coupons in urban areas (economic barrier)
\end{outline}

\paragraph{Political Dominance and Instability} Mao believed that politics overweight economics $\to$ a series of strong investment growth followed by strong reductions (economic instability)
\begin{table}[ht]
\centering
\caption{Mao's political targets.}
\begin{tabular}[t]{lc}
\toprule
&Targets\\
\midrule
1957&Rightists\\
1959&Rightist factions within the Party\\
1966&Bureaucratic establishment within the Party\\
1971&Lin Biao (his successor)\\
1975&Zhou Enlai (Premier) and Confucius\\
\bottomrule
\end{tabular}
\end{table}%

\subsubsection{Major Events}
\paragraph{1949-52}
\begin{outline}[enumerate]
\1 Land Reform---redistribution of land to peasants
\1 Economic recovery with industrial investments (mostly) in the Northeast
\1 Expropriation of Japanese factories
\1 Korean War (1950-53)
\end{outline}

\paragraph{1953-56 (First Five Year Plan)}
\begin{outline}[enumerate]
\1 Import of Soviet planning system, advisers, and technology
\1 Strong investment push in 1953 $\to$ inflationary pressure (shortages)
\1 Collectivization in the country side and nationalization in urban areas (1955-56) $\to$ peak (Exaggerated) investment in 1956
\end{outline}

\paragraph{1956-57 ("Hundred Flowers")}
"Hundred Flowers" campaign encouraged freedom of expression in discussing China's future.
\begin{outline}[enumerate]
\1 Background: 20th congress in USSR recognized new "national paths" to socialism and denounced Stalin (1956)
\1 Partial economic re-liberalization and increased role to market and private sector (correction relative to 1956)
\end{outline}

\paragraph{1958-60 (Great Leap Forward)}
\begin{outline}[enumerate]
Divergence from Soviet model $\to$ Maoist economic plan (mobilization) $\to$ "Great Leap Forward" based on idea of fast leap towards communism (aim to surpass Britain in 3 years)
\1 Launch of "anti-rightist" campaign against intellectuals who had expressed options during "Hundred Flowers"
\1 Establishment of communes in the countryside, abolition of material incentives, and elimination of markets in the countryside $\to$ encouraged communal living
	\2 Economic decentralization and encouragement of local technologies (e.g., local furnaces)
\1 Substantial increase in grain procurement + inflated growth reports $\to$ nearly all outputs went to the city $\to$ large rural famine in 1960 (largest in 20th century; $\geq 30$ million deaths)
	\2 Fanaticism within CCP prevented correction $\to$ Peng Dehuai repressed for criticizing and caused the launch of a new anti-rightist campaign within CCP
	\2 Correction in 1961-63
\end{outline}
\begin{remark}
Similar issue in USSR regarding Ukraine
\end{remark}

\paragraph{1966-1976 (Cultural Revolution)}
\begin{outline}[enumerate]
\1 Mao encouraged Ref Guards to attack CCP and overthrow his rivals (e.g., Liu Shaoqi, Deng Xiaoping)
\1 Economically: militarization (PLA in economy); decentralization; autarky; no material incentives
\1 Universities closed; intellectuals and young people sent to the countryside
\1 After Mao's death, Gang of Four arrested, interim by Hua Guofeng until serious reforms decided in 1978
\end{outline}
\clearpage

\section{A Picture of the Reform Process}
\subsection{Initial Conditions}
\begin{outline}[enumerate]
Low GDP per capita in 1978 $\to$ uninterrupted growth since 1979 (unprecedented since 1840) $\to$ \textit{ex-ante} doubts: economists did not anticipate China's growth; political scientists predicted political stabilization before economic stabilization
\end{outline}

\subsection{Difference in Reform Strategies}
\begin{table}[ht]
\centering
\caption{Reform strategy.}
\begin{tabular}[t]{cc}
\toprule
China&Easter Europe\\
\midrule
Gradualism&Big Bang\\
Incremental&Comprehensive\\
Prioritize economic reform&Prioritize political reform\\
\bottomrule
\end{tabular}
\end{table}%

\subsection{1979-1993 (First Stage; Growing Out of the Plan)}
\begin{remark}
Many of the evolution traits were unplanned.
\end{remark}
\subsubsection{1979-83 (Phase I; The start of reforms)} The 3rd Plenum of the 11th CCP Congress (December, 1978) designated economic development as the key task with key phrases ``reform \& open up" and frame works of ``planning as the principle part and market as a supplementary part" $\to$ reforms in agriculture, foreign investment, special economic zones, profit incentives, fiscal decentralization, etc.
\begin{outline}[enumerate]
\1 Decollectivization
	\2 Spontaneous disbanding of communes + household responsibility system (first experimented in Sichuan; later generalized) $\to$ basic work unit became the household
	\2 Households received a 15-year lease on land (strong investment incentive), delivering quotas the the state and freedom for residual production $\to$ huge increase in agriculture output
	\2 Family members that do not need to work the land had the freedom to work in township-village enterprises (additional income to the household)
\1 Fiscal Decentralization---provinces signed contracts with the central government to hand over fixed sum, tax revenues, or share to central government $\to$ incentive to increase the tax base and to maximize growth (similar to household responsibility system)
\end{outline}

\subsubsection{1984-88 (Phase II; high wave of reforms)} October 1948 saw major decisions on urban reform (new framework of ``planned commodity economy'') $\to$ market price liberalization; managerial incentives in SOEs; coastal open cities and development zones; financial reforms (e.g., commercial banks)
\begin{outline}[enumerate]
\1 Agricultural success---grain production increased from 319 kg to 400 kg (per capita) between 1978 and 1984; rural income (per capita) increased by more than 50\%
	\2 Boost in confidence and popular support for reformers
\1 Dual-Track Liberalization
	\2 Plan track---fulfillment of plan contract (planned quantities and prices)
	\2 Market track---freedom at the margin to sell residual quantity at free market prices $\to$ price liberalization at the margin
	\2 Achieves the same efficiency as a full market liberalization while avoiding harm to any economic agents (consumer and producer) and production disruption (as was the case in Eastern Europe)
\1 Special Economic Zones---Shenzheng, Zhuhai, Shantou, Xiamen, and Hainan (list later extended)
	\2 The initial goal was to set up enterprises with the help of private/foreign capital that produced only for export (will not disturb the domestic economy)
\end{outline}

\subsubsection{1989-93 (Phase III; retreat and revival of reforms)}
\begin{outline}[enumerate]
\1 Problems of inflation and corruption + Tiananmen Square incident (June 4, 1989)
\1 June 4, 1989: Tiananmen square
\1 Central government retreat of reform, but some local governments continues reform, especially in coastal provinces
\1 January and February of 1992: Deng Xiaoping's "Southern Tour" $\to$ beginning of the revival
\end{outline}

\subsection{1994-2001 (Second Stage; Building the Market System)}
\begin{outline}[enumerate]
\1 September 1992: ``socialist market economy''
\1 November 1993: a blueprint for building a market system
\1 January 1, 1994: foreign exchange reform; tax and fiscal system reform; monetary and financial reform; social safety net; privatization of small SOEs
\1 March 1999: constitutional amendments
	\2 ``Private economy is a supplement to public ownership'' $\to$ ``private ownership is an important component of the economy'' (on private ownership)
	\2 ``Governing the country according to law'' (on rule of law)
\end{outline}

\begin{table}[ht]
\centering
\caption{Contrasting styles of economic reform.}
\begin{adjustbox}{width={\textwidth}}
\begin{tabular}[t]{p{.5\textwidth}p{.5\textwidth}}
\toprule
1980s reform&1990s reform\\
\midrule
Zhao Ziyang: cautious, consensual decision-making&Zhu Rongji: rapid, personalized decision-making\\
Introducing markets where feasible; focus on agriculture and industry&Strengthen institutions of market economy; focus on finance and regulation\\
Dual-track strategy&Market unification, unite dual tracks\\
Particularistic contracts with powerful incentives&Uniform rules: ``level playing field''\\
Competition created by entry; no privatization&State-sector downsizing; beginnings of privatization\\
Decentralize authority and resources&Recentralize resources, macroeconomic control\\
Inflationary economy with shortages&Price stability, goods in surplus\\
``Reform without losers''&Reform with losers\\
\bottomrule
\end{tabular}
\end{adjustbox}
\end{table}%

\subsection{2002-2012 (Third Stage; Integrating to the Global Economy)}
No major institutional reforms under Hu Jintao

\begin{outline}[enumerate]
\1 December 11, 2001: China entered the WTO
\1 2002-2006: a five-year window period for transition $\to$ accelerated growth between 2003 and 2006
\1 A lot of infrastructure investment (e.g., roads, highways, railroads) $\to$ emphasis on solving economic inequalities in inland provinces, which grew much slower than coastal provinces
	\2 Might also have the intention of easing alienation among minorities
\1 November 2002: 16th CCP Congress decided that capitalists/entrepreneurs can join the party (also peaceful change in leadership)
\1 March 2004: constitutional amendment on private property rights---``the lawful private property of citizens is not to be violated''
\1 Increased inequality, corruption, social unrest
\1 Slogans: "harmonious society" and "peaceful development"
\1 New technology: the Internet and the web; cell phones; satellite and cable TV
\1 Continued crackdown on media, but allowed minimum freedom of expression on social media, which may lead to policy shifts $\to$ media's role in the rise of populism
\end{outline}

\subsection{2012-Present (Forth Stage)}
The third plenum (November 2013) stated that the market should play a ``decisive role'' in the economy $\to$ predicted reform wave, but in practice the party placed emphasis on anti-corruption and reinforcement of the party and the state
\begin{outline}[enumerate]
\1 Reform of the ``one child policy''
\1 Change in spirit after 2008 crisis---potential switch to a ``Chinese model'' with large state sector, tighter party control over enterprises, and rein in expenses of bureaucracy
\1 Crackdown on Ant Group and Didi
\1 Regulation of fintech but also data protection
\1 ``Common Prosperity''---the rich are asked to help
\1 Cultural crackdown on movie stars and pop music stars, gaming, ``effeminate'' boys
\end{outline}
\clearpage

\section{Specificity of Chinese Institutions}
\begin{outline}[enumerate]
\1 Party-government duality---party organs above all government organs
\1 Mixture of institutional innovation (Leninist power structure) and historical institutional tradition
\end{outline}

\subsection{History of Communist Organizational Form}
Tsarist regime with powerful secret police good at dismantling revolutionary organizations $\to$ need for an organizational form that can survive repression, work in illegal conditions, and capable of seizing power by violent means (Leninist organizational doctrine) $\to$ powerful military-esque organizations with large use of ideology for disciplining purposes

\subsubsection{Principles of Leninist Organization}
\begin{outline}[enumerate]
\1 Strict conditions for party membership---members are evaluated regularly, possibly for promotion or exclusion; lack of private sphere
\1 Democratic centralism
	\2 Members may only express personal views on political issues within their unit (party cell)---avoid fractionism
	\2 Party cell membership restricted to work unit with self-contained capacity to act---no horizontal communication with members of other units
	\2 Strict organizational hierarchy with vertical flows of information and of party command
		\3 Politburo is the head of organization---daily meetings ensures fast actions
\1 Supreme organs
	\2 Party Congress---composed of delegates elected from below; decides Party program and statutes for future and elects Central Committee
	\2 Central committee---meets regularly between Congresses to decide on party orientations and elects top leaders (similar to a party Parliament)
	\2 Politburo---\textit{de facto} leader; each politburo member is in charge of a specialized task; control over the organizational machine (everyone below central committee) assures more power (e.g. Stalin)
\end{outline}

\subsubsection{Characteristics of Leninist Organization}
\begin{outline}[enumerate]
\1 Leninist organization is an elite organization designed for united action as well as efficiency in action $\to$ the organizational machine focused on action, not debate
\1 Difficulty challenging incumbent leaders in absence of a serious crisis without accusations of fractionism, breach of statutes, and/or ideological deviationism
\1 Strong working record matters more than other skills in promotion
\end{outline}

\subsection{Erosion of Communist Power under Planning Economy}
The communist party manages both political and economic power $\to$ burden of managing an economy with no real balanced central plan and with economic complexity without using the market as a central resource allocation mechanism $\to$ economic stagnation + erosion of power
\begin{outline}[enumerate]
\1 Step 1: replacement of mandatory planning by some form of non-binding plan for enterprises
\1 Step 2: increase in decision-making autonomy of managers (in particular on prices and wages)
\1 Step 3: privatization of assets to managers
\1 Step 4: implosion of communist power structure + power race among political entrepreneur $\to$ emergence of oligarchs and kleptocracy
\end{outline}

\subsection{The Chinese Institutional Innovation}
Transition to a market economy with mostly private ownership within the communist regime with the goal of preserving and consolidating the power of CPC $\to$ CPC's power over the market economy makes it one of the most powerful organizations in all of history
\begin{outline}[enumerate]
\1 Private entrepreneurs can become CPC members since 2001
\1 No political reform following economic reform
\1 Growth objectives were pursued using existing CPC institutions + market forces relied on government decentralization and yardstick competition (meritocracy) $\to$ reinforced power of CPC in all spheres
\end{outline}

\subsection{The Structure of CPC}
\begin{outline}[enumerate]
\1 Central Committee (CC) with 205 full members and 171 alternate (non-voting) members
	\2 Members are provincial party secretaries and governors, ministers and minister-ranked, PLA head
	\2 9-year term of office
	\2 Mandatory retirement at 65
	\2 62\% turnover rate at Party Congresses $\to$ average number of mandates less than 3
\1 Politburo with 25 members---not as powerful due to role of PSC
	\2 Members are party chiefs from important provinces, heads of military, of government, and of important commissions of the Central Committee
	\2 Meets once a month
\1 Politburo Standing Committee (PSC)---center of power
	\2 Meets once a week
	\2 7 members: secretary general Xi Jinping, premier Li Keqiang, Li Zhanshu, Wang Yang, Wang Huning, Zhao Leji, Han Zheng
		\3 3 supporters of Xi: Li Zhanshu, Zhao Lezhi, Wang Huning
		\3 2 communist youth league background: Li Keqiang, Wang Yang
		\3 1 Shanghai faction: Han Zheng
\end{outline}

\subsection{The Power of CPC}
CPC has power over a large number of people, inside and outside China, and can affect a very large set of decisions of individuals; inside China, the power of CPC permeates all layers of society.
\begin{outline}[enumerate]
\1 CPC transitioned from totalitarian to authoritarian (no thought policing)
\1 CPC controls government at all levels, but also has power over private enterprises, schools, universities, law firms, mass movements, public organizations, and NGOs
	\2 Principle of ``leading role of the CPC'' at all levels of society $\to$ party holds official truth
	\2 CPC is by doctrine ``above the law,'' but makes and uses the law
	\2 Leninist organizational apparatus (nomenklatura system)---organization department has files on careers of all cadres, and party, at all levels, follows careers of officials and controls appointments and promotions
		\3 Performance evaluation (in function of party's current goals) important for promotion
		\3 Party control is an obstacle to standard corporate governance in firms
\1 Total control over PLA---the military is not allowed to make public statements
\1 Control over media through propaganda department and censorship---censorship prevents collective action and ``showing weakness''
\1 CPC positions carry prestige and offer possibilities of social mobility; however, all must start at the bottom to acquire experience of power at all levels
\end{outline}


\subsection{CPC through the Lens of Political Science}
\subsubsection{Succession Problem}
\begin{outline}[enumerate]
\1 Mandatory retirement age of 65 for CC
\1 Must be age 50-68 to be nominated for PSC---restricts list of possible nominees
\1 Current leader chooses successor's successor (Xi is unlikely to follow succession rules)
\1 Bo Xilai incident---Bo wanted to use his influence to make a bid for power but failed
\end{outline}

\subsubsection{Information Problem}
While top party leaders have always used vertical channels of information (reporting duties important at all levels), leaders have learned to rely on alternative sources of information (e.g., tolerance of local revolts, whistle-blowing, social media) in the event vertical channels fail (e.g., Great Leap Forward)

\subsubsection{Taxation Problem}
\begin{outline}[enumerate]
\1 Collectivist culture relies on extended family to provide social insurance $\to$ despite strong state capacity giving high capacity to tax, China has traditionally set taxes and standard government expenditures at relatively low level
\1 Significant state ownership gives direct control over state resources $\to$ helps keep tax rates low and reduces tax distortions
\1 Mobilization and campaigns are a standard instrument of CPC that can be used in exceptional times
\end{outline}

\subsubsection{Autocratic Trade-Off Problem}
In typical autocratic systems, loyalty and competence are at odds; however, the party-state power duality in China solves this fundamental trade-off by separating political and mobilization power (provincial party secretary) and economic power and sense of benevolence (provincial governor) $\to$ the population cannot infer the degree of benevolence of the secretary $\to$ lack of support for popular revolt $\to$ better promotion of meritocracy.

\begin{remark}
This is currently the best (albeit unintentional) solution to the loyalty-competence trade-off under autocracy, providing strong regime stability in China.
\end{remark}
\clearpage

\section{Growth and Structural Change}
\subsection{Possible Problems with Chinese Growth Data}
While growth is real, there are uncertainties with the official growth data:
\begin{outline}[enumerate]
\1 Miscalculation of the GDP deflator---growth rates could be overstated by 1-2\%
\1 Reporting incentives for exaggeration by local officials
\1 Missing data from small-scale, private firms $\to$ 1998 data collection reform implied sample survey of small firms
\1 Expanding scope of products
\end{outline}

\subsection{Economic Growth Patterns}
Growth rate is negatively correlated with the level of per capita income (law of diminishing returns). China's growth is similar to other high-performing East Asian economies (i.e., not the fastest growing economy), and is projected to slow down (other East Asian economies saw slow down after reaching middle income level)
\begin{outline}[enumerate]
\1 Getting incentives right
\1 Making the market work
\1 Opening up the economy
\end{outline}


\subsection{Special Factors in China}
\begin{outline}[enumerate]
\1 Size---population is close to three times the 8 high performing East Asia economies combined
	\2 Approximately 20\% of world total
	\2 Benefit: Non-linear economic scaling
\1 Openness---China is unusually open among large economies with the largest imports and exports
	\2 Benefit: markets; technologies; competition
\1 Institutions---an emerging market economy with central planning legacy (more government intervention than other East Asian economies)
	\2 Unfinished economic reforms
	\2 May delay structural adjustments $\to$ slow moving from imitation to innovation
\end{outline}

\subsection{Source of Growth}
\subsubsection{Theories of Growth}
\begin{outline}[enumerate]
\1 Classical---physical capital; human capital; labor; technical progress
\1 Extended---geography; opening; institutions; culture
\end{outline}
\subsubsection{Growth Accounting in China}
\begin{outline}[enumerate]
\1 Capital accumulation has been the most important factor for growth, contributing about 62\% of growth
\1 Labor reallocation from agriculture to industry contributes 10\% of growth
\1 TFP (total factor productivity; depends on reform and innovation) growth contributes 28\% of growth
\end{outline}

\subsection{Three Sectors}
\begin{outline}[enumerate]
\1 Primary Sector---agriculture
\1 Secondary Sector---manufacturing, mining, utilities, construction
\1 Tertiary (Service) Sector---restaurants, trade, transportation, telecommunications, financial services, real estate, education, health care, government administration, etc.
\end{outline}

\subsection{Structural Issues in China}
\begin{outline}[enumerate]
\1 Sectoral changes, labor reallocation, and urban-rural shift
	\2 Labor moving from the agricultural sector to the non-agricultural sector
	\2 Labor moving from the domestically oriented sector to the export oriented manufacturing sector
	\2 Labor moving to the service sector
\1 High level of investment with decreasing return (mostly invested in secondary sector)
\1 Distortions in measuring structure change using distorted prices:
	\2 Low agricultural prices underestimate the share of primary sector and high industrial prices overestimate the share of secondary sector
\end{outline}

\section{Agriculture}
\subsection{Background Information}
3.3 million natural villages in China, but many villages merged with the urban areas, and other dwindled to small settlements (single-lineage villages) $\to$ strong inequality between villages.

\begin{remark}
Rural area is not the same as the agriculture sector, for many worked in urban areas or in rural (without urban permit), non-agricultural industries.
\end{remark}

\subsection{Significance of the Agriculture Sector}
\begin{outline}[enumerate]
\1 The traditional sector that had the largest employment
\1 The disaster sector under central planning
\1 Sector that saw the first reform success
\1 Economic linkages to other sectors
	\2 Releasing excess labor 
	\2 Providing savings 
	\2 Generating consumer demand
\end{outline}

\subsection{Organization of Agriculture}
Agricultural collectives (1956-58; 100-250 households) $\to$ communes (1958-59; over 5,000 households) $\to$ teams (1962-81; approx. 30 households) $\to$ households (1982-present)

\subsubsection{Agricultural Collectives}
The land was pooled/``collectively owned'' and worked in common, with the collective serving as the basic accounting unit $\to$ net income was distributed to households via accumulation of ``work points''
\begin{outline}[enumerate]
\1 Distribution of income (post-taxation and public funds retention) to households by the year end in grain and cash
\end{outline}

\paragraph{Pros}
\begin{outline}[enumerate]
\1 Achieve the goal of prioritizing grain production in central planning
\1 Mobilization of resources (lands) for big projects (e.g., irrigation)
\1 Social benefits---water supply, education, health care, some insurance functions (e.g., risk pooling)
\1 Rural industries
\end{outline}
\paragraph{Cons}
\begin{outline}[enumerate]
\1 Very inefficient in agricultural production due to lack of incentives (work points only measures time spent, but not effort)
\1 Difficulty making adjustments according to age, gender, and health (manual work)
\end{outline}

\paragraph{Consequences}
\begin{outline}[enumerate]
\1 Low grain production growth, and even lower non-grain agriculture products growth (per capita growth rate below zero)
\1 Improved social indicators---decreased infant mortality, increased life expectancy, increased rate of literacy, and increased rural infrastructure
\end{outline}

\subsubsection{Household Responsibility System (Agricultural Reform)}
Agriculture reform that grants better incentives for production. Started as an experiment by peasants in Anhui, beginning in 1978. Fully encouraged by the central government by 1982.
\begin{outline}[enumerate]
\1 Form 1: linking the remuneration of a small group or household to the output of a specific plot of land (sharecropping)
\1 Form 2: an individual household pays a fixed amount to the government and keeps the rest (individual household farming; residual claimant)
\1 Land owned by the village collective and cannot be used as collateral in loans
\1 Redistribution of land to accommodate demographic changes
\end{outline}

\begin{remark}
Trade-off between incentives and risk sharing (good year vs. bad year). Form 1 has more risk sharing with the government.
\end{remark}

\paragraph{Side-effects}
Decline of some rural public services
\begin{outline}[enumerate]
\1 Health care---barefoot doctors disappeared, hospital beds stagnated, village and township paramedics declined, private facilities increased but expensive
\1 Elementary school---despite nationwide 9 year compulsory education, rural education is in trouble
	\2 Supply side: local public finance constraint
	\2 Demand side: school drop out due to current economic concerns
\end{outline}

\subsection{Agricultural Economics}
\begin{outline}[enumerate]
\1 Demand
	\2 Calories and protein came mainly from grain
	\2 Changing diet: shifting from grain to meat, vegetables, fruits, fish, poultry, etc.
	\2 Forecasting future demand is critical for the world market
\1 Supply
	\2 Grain production is land intensive---China is a land scarce country
	\2 Meat production requires a high conversion ratio of grains---4:1 for port and 2:1 for poultry
	\2 Comparative advantage of labor intensive products (e.g., vegetables, fruits)
\1 Government policies
	\2 Before 1996---agriculture product prices below world prices $\to$ redistribution from rural to urban
	\2 After 1996---agriculture product prices above world prices $\to$ redistribution from urban to rural
\1 Import (wheat and maize) + export (rice) $\to$ net importer
	\2 Agricultural tariffs for wheat and maize
\1 Net self sufficiency rate in grain = 95\%
\end{outline}

\subsection{Agricultural Technology Change}
\begin{outline}[enumerate]
\1 Expanding usable land through intensification
	\2 Land improvement through ``terracing''
	\2 Multi-cropping: rotation, intercropping, and/or relay cropping
\1 Changing crops
	\2 Grain crops: rice, wheat, maize, potatoes, etc.
	\2 Economic or cash crops: Cotton, oil seeds, sugar crops, tobacco, vegetables, fruits, etc.
\end{outline}

\subsubsection{Green Revolution}
Creation of conditions for growth of high yield varieties
\begin{outline}[enumerate]
\1 Improved seeds---hybrid rice and wheat from R\&D in multi-level research facilities
	\2 Yuan Longping---father of hybrid rice (1929-2021)
\1 Fertilizer---domestic production + imports
\1 Water---irrigation facilities [depending on regional need; Rice growing region (south of Huai river) vs. wheat growing region (north of Huai river)]
\end{outline}

\subsubsection{Rural Motive Power}
Sharp increase in mechanical power post-reform: hydraulic machinery (22\%), small tractors (22\%), transport equipment (17\%), processing equipment (13\%), and others (26\%)

\subsection{Sources of Agriculture Growth}
\begin{outline}[enumerate]
\1 Inputs (land, labor, capital)
\1 Productivity
	\2 Institutional change (household responsibility system)
	\2 price change (increase of procurement prices for grain)
	\2 Technical innovation
		\3 Using different types of inputs (seeds, fertilizers, irrigation)
		\3 Shifting cropping patterns (multiple cropping, non-grain crops)
		\3 Change to other high value products (off season vegetables, organic products)
\end{outline}

\subsubsection{Agriculture Responses}
\begin{outline}[enumerate]
\1 1978-84
	\2 Total crop output growth: 42\%
	\2 Household responsibility system productivity growth: 18\% (or 42\% of growth)
	\2 State procurement price increase: 7\% (or 16\% of growth)
\1 Shifts in demand---reduction in demand for sorghum and millet
	\2 Household demand for grain has gone down, but demand for grain has remained strong because of fodder to feed animals
	\2  Meat industry has expanded rapidly, not always with good monitoring
\end{outline}

\subsection{Agricultural Policy}
Agricultural output has diminishing returns but can be influenced by policy through instruments affecting revenues and costs.
\subsubsection{Agricultural Policy Life Cycle}
Initially agriculture is taxed (and main source of taxation), but later governments tend to subsidize agriculture (true everywhere)
\begin{outline}[enumerate]
\1 Until 1994, prices to farmers were much lower than international prices.
\1 Price differential disappeared in 1994-2005. 
\1 In 2007-2008, rise in global food prices, policy-makers kept domestic prices down and prohibited food exports.
\1 In 2014, fall in world prices but support of domestic prices and begin of a subsidy era.
\subsubsection{Specific Policies}
Agricultural policies are costly and lead to excess grain production (some of it left to rot), but protect against international disruption and excess land speculation.
\1 In 2005, abolition of agricultural tax (5-7\% of value added) and subsidy program:
	\2 Direct payments and input subsidies for farmers (2.4\% of value added in 2012).
	\2 Specific project and input subsidies (4\% of value added in 2011).
	\2 Support prices for main commodities.
\1 Land use: In 2006, target of keeping 120 mlnha of cultivated land. Plan disaggregated by province. Attempt to counter local authorities who convert land for development.
\1 Policy of self-sufficiency in grain (95\% for staple cereals, not for non food grain). Rules designed to be within WTO rules.
	\2 NOTE: China imports very little food but large amounts of natural resources for industry.
\end{outline}

\section{State Industry and State-Owned Enterprises}

\section{Non-State Industry}

\section{The Financial Sector}

\section{Population, Labor, and Income Distribution}

\section{Foreign Trade and Investment}

\section{China in the Twenty-First Century}
\end{document}


